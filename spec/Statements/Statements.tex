\index{statement}

Chapel is an imperative language with statements that may have side
effects.  Statements allow for the sequencing of program execution.
They are as follows:
\begin{syntax}
statement:
  block-statement
  expression-statement
  conditional-statement
  select-statement
  while-do-statement
  do-while-statement
  for-statement
  return-statement
  yield-statement
  module-declaration-statement
  function-declaration-statement
  type-declaration-statement
  variable-declaration-statement
  use-statement
  type-select-statement
  empty-statement
  cobegin-statement
  begin-statement
  serial-statement
  atomic-statement
  on-statement
\end{syntax}

The declaration statements are discussed in the sections that define
what they declare.  Module declaration statements are defined
in~\rsec{Modules}.  Function declaration statements are defined
in~\rsec{Functions}.  Type declaration statements are defined
in~\rsec{Types}.  Variable declaration statements are defined
in~\rsec{Variables}.  Return statements are defined
in~\rsec{The_Return_Statement}.  Yield statements are defined
in~\rsec{The_Yield_Statement}.

The \sntx{cobegin-statement} is defined in~\rsec{Cobegin}.
The \sntx{begin-statement} is defined in~\rsec{Begin}.
The \sntx{serial-statement} is defined in~\rsec{Serial}.
The \sntx{atomic-statement} is defined in~\rsec{Atomic_Transactions}.
The \sntx{on-statement} is defined in~\rsec{On}.
