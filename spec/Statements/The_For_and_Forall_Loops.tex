The for and forall loops iterate over sequences, domains, arrays,
iterators, or any class that implements the structural iterator
interface.  The syntax of the for and forall loops are given by:
\begin{syntax}
for-statement:
  for-tag index-list `in' iterator-list `do' statement
  for-tag index-list `in' iterator-list block-level-statement

for-tag:
  `for'
  `forall'
  `ordered' `forall'

index-list:
  expression [[,...]]

iterator-list:
  expression [[,...]]
  ( expression [[,...]] )
  [ expression [[,...]] ]
\end{syntax}

When there is single expression in the iterator list, that expression
is iterated over, and on each iteration, the value returned by the
iterator is assigned to the expressions in the index list and the
statement is executed.

Where there are multiple expressions in the iterator list, if they are
surrounded in parentheses or unsurrounded, the expressions are
iterated over using a zipper iteration described in
Section~\ref{Zipper_Iteration}.  If the expressions are surrounded by
brackets, they are iterated over using a tensor product iteration
described in Section~\ref{Tensor_Product_Iteration}.

The \chpl{forall} and \chpl{ordered forall} statements have parallel
semantics that are described in Sections~\ref{Forall}
and~\ref{Ordered_Forall}.
