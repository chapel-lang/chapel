A type select statement has two uses.  It can be used to determine the
type of a union, as discussed in
Section~\ref{The_Type_Select_Statement_and_Unions}.  In its more
general form, it can be used to determine the types of one or more
values using the same mechanisms used to disambiguate function
definitions.  It syntax is given by:
\begin{syntax}
type-select-statement:
  `type' `select' expression-list { type-when-statements }

type-when-statements:
  type-when-statement
  type-when-statement type-when-statements

type-when-statement:
  `when' type-list `do' statement
  `when' type-list block-level-statement
  `otherwise' statement

expression-list:
  expression
  expression , expression-list

type-list:
  type
  type , type-list
\end{syntax}

Call the expressions following the keyword \chpl{select}, the select
expressions.  The number of select expressions must be equal to the
number of types following each of the \chpl{when} keywords.  Like the
select statement, one of the statements associated with a \chpl{when}
will be executed.  In this case, that statement is chosen by the
function resolution mechanism.  The select expressions are the actual
arguments, the types following the \chpl{when} keywords are the types
of the formal arguments for different anonymous functions.  The
function that would be selected by function resolution determines the
statement that is executed.  If none of the functions are chosen, the
the statement associated with the keyword \chpl{otherwise}, if it
exists, will be selected.

As with function resolution, this can result in an ambiguous
situation.  Unlike with function resolution, in the event of an
ambiguity, the first statement in the list of when statements is
chosen.
