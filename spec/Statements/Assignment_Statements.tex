An assignment statement assigns the value of an expression to another
expression that can appear on the left-hand side of the operator, for
example, a variable.  Assignment statements are given by

\begin{syntax}
assignment-statement:
  left-hand-side-expression assignment-operator expression

left-hand-side-expression:
  expression

assignment-operator: one of
   = += -= *= /= %= **= &= |= ^= &&= ||= #= <<= >>=
\end{syntax}

The expression on the right-hand side of the assignment operator is
evaluated first; it can be any expression.  The expression on the left
hand side is any expression that can be assigned the value on the
right-hand side.  It is evaluated second and then assigned the value.

The assignment operators that contain a binary operator as a prefix is
a short-hand for applying the binary operator to the two operands and
then assigning the value of that application to the already evaluated
left-hand side.  Thus, for example, \chpl{x += y} is equivalent to
\chpl{x = x + y} where the expression \chpl{x} is evaluated once.

Values of one primitive or enumerated type can be assigned to another
primitive or enumerated type if an implicit coercion exists between
those types~\ref{Implicit_Conversions}.

The validity and semantics of assigning between
classes~\ref{Class_Assignment}, records~\ref{Record_Assignment},
unions~\ref{Union_Assignment}, tuples~\ref{Tuple_Assignment},
sequences~\ref{Sequence_Assignment}, domains~\ref{Domain_Assignment},
and arrays~\ref{Array_Assignment} is discussed in these later
sections.
