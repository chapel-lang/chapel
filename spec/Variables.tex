\sekshun{Variables}
\label{Variables}

A variable is a symbol that represents memory.  Chapel is a
statically-typed, type-safe language so every variable has a type that
is known at compile-time and the compiler enforces that values
assigned to the variable can be stored in that variable as specified
by the type.

\subsection{Variable Declarations}
\label{Variable_Declarations}

Variables are declared with the following syntax:
\begin{syntax}
variable-declaration-statement:
  `config'[OPT] variable-kind variable-declaration ;

variable-kind: one of
  `param' `const' `var'

variable-declaration-list:
  variable-declaration
  variable-declaration , variable-declaration-list

variable-declaration:
  identifier-list type-part[OPT] initialization-part
  identifier-list type

identifier-list:
  identifier
  identifier , identifier-list

type-part:
  : type

initialization-part:
  = expression
\end{syntax}
A \sntx{variable-declaration-statement} is used to define one or more
variables.  If the statement is a top-level module statement, the
variables are global; otherwise they are local.  Global variables are
discussed in~\rsec{Global_Variables}.  Local variables are discussed
in~\rsec{Local_Variables}.

The optional keyword \chpl{config} specifies that the variables are
configuration variables, described in
Section~\rsec{Configuration_Variables}.

The \sntx{variable-kind} specifies whether the variables are
parameters (\chpl{param}), constants (\chpl{const}), or regular
variables (\chpl{var}).  Parameters are compile-time constants whereas
constants are runtime constants.  Both levels of constants are
discussed in~\rsec{Constants}.

Multiple variables can be defined in the same variable declaration
list.  All variables defined in the same \sntx{identifier-list} are
defined to have the same type and initialization expression.

The \sntx{type-part} of a variable declaration specifies the type of
the variable.  It is optional if the \sntx{initialization-part} is
specified.  If the \sntx{type-part} is omitted, the type of the
variable is inferred using local type inference described
in~\rsec{Local_Type_Inference}.

The \sntx{initialization-part} of a variable declaration specifies an
initial expression to assign to the variable.  If
the \sntx{initialization-part} is omitted, the variable is initialized
to a default value described in~\rsec{Default_Initialization}.

\subsubsection{Default Initialization}
\label{Default_Initialization}

If there is no initialization, a variable is initialized to the
default value of its type.  The default values are as follows:
\begin{itemize}
\item For numeric types, the default value is zero.
\item For \chpl{bool} type, the default value is \chpl{false}.
\item For \chpl{string} type, the default value is \chpl{""}.
\item For enumerated types, the default value is the first enumeration constant.
\item For class types, the default value is \chpl{nil}.
\item For record types, the default value is defined to be a newly constructed record.
\end{itemize}

\subsubsection{Local Type Inference}
\label{Local_Type_Inference}

This is a stub.  This portion of the document does not exist.

\subsubsection{Multiple Variable Declarations}
\label{Multiple_Variables}

\subsection{Global Variables}
\label{Global_Variables}

This is a stub.  This portion of the document does not exist.

\subsection{Local Variables}
\label{Local_Variables}

This is a stub.  This portion of the document does not exist.

\subsection{Constants}
\label{Constants}

This is a stub.  This portion of the document does not exist.

\subsubsection{Compile-Time Constants}
\label{Compile-Time_Constants}

This is a stub.  This portion of the document does not exist.

\subsubsection{Runtime Constants}
\label{Runtime_Constants}

This is a stub.  This portion of the document does not exist.

\subsection{Configuration Variables}
\label{Configuration_Variables}

This is a stub.  This portion of the document does not exist.
