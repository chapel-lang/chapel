\documentclass[10pt,twoside,titlepage]{article}
\usepackage{hyperref}
\usepackage{graphicx}
\usepackage{listings}
\lstdefinelanguage{chapel}
  {
    morekeywords={
      atomic,
      begin, bool, break, by,
      class, cobegin, coforall, compilerError, compilerWarning, complex, config, const, continue,
      def, delete, distributed, do, domain,
      else, enum,
      false, for, forall,
      goto,
      if, imag, in, index, int, inout,
      label, let, local, locale,
      module,
      new, nil,
      on, opaque, otherwise, out,
      param,
      range, real, record, reduce, return,
      scan, select, serial, single, sparse, string, subdomain, sync,
      then, true, tuple, type,
      uint, union, use,
      var,
      when, where, while,
      yield
    },
    sensitive=false,
    mathescape=true,
    morecomment=[l]{//},
    morecomment=[s]{/*}{*/},
    morestring=[b]",
}

\lstset{
    basicstyle=\footnotesize\ttfamily,
    keywordstyle=\bfseries,
    commentstyle=\em,
    showstringspaces=false,
    flexiblecolumns=false,
    numbers=left,
    numbersep=5pt,
    numberstyle=\tiny,
    numberblanklines=false,
    stepnumber=0,
    escapeinside={(*}{*)},
    language=chapel,
  }

%\newcommand{\chpl}[1]{\lstinline[language=chapel,basicstyle=\ttfamily,keywordstyle=\bfseries]!#1!}
\newcommand{\chpl}[1]{\lstinline[language=chapel,basicstyle=\small\ttfamily,keywordstyle=]!#1!}
\newcommand{\varname}[1]{\emph{#1}}
\newcommand{\typename}[1]{\emph{#1}}
\newcommand{\fnname}[1]{\chpl{#1}}

\lstnewenvironment{chapel}{\lstset{language=chapel,xleftmargin=2pc,stepnumber=0}}{}
\lstnewenvironment{invisible}{\lstset{language=chapel,xleftmargin=2pc,stepnumber=0,keywordstyle=\bfseries\color{white}}}{}
\lstnewenvironment{chapel0}{\lstset{language=chapel,stepnumber=0}}{}

\lstnewenvironment{numberedchapel}{\lstset{language=chapel,xleftmargin=15pt,stepnumber=1}}{}

\lstnewenvironment{chapelcode}{\lstset{language=chapel,stepnumber=1}}{}

\lstnewenvironment{commandline}{\lstset{keywordstyle=,xleftmargin=2pc}}{}

\lstnewenvironment{protohead}{\lstset{language=chapel,xleftmargin=0pc,belowskip=-10pt,stepnumber=0}}{}

\newenvironment{protobody}{\begin{description}\item[\quad\quad] }{\end{description}}

\lstdefinelanguage{syntax}
  {
    morekeywords={
      block,
      conditional,
      expression,
      level,
      statement
    },
    sensitive=false,
    mathescape=false,
    basicstyle=\footnotesize\tt,
    keywordstyle=\em,
    literate={[[}{{$[$}}{1}
             {]]}{{$]$}}{1}
             {[[,...]]}{{$[$, $\ldots]$}}{5}
             {[[...]]}{{$[\ldots]$}}{3}
  }

\lstnewenvironment{syntax}{\lstset{language=syntax,xleftmargin=2pc}}{}

% block
% block-level-statement
% conditional-statement
% expression
% statement


\newenvironment{example}{
\begin{center}
\begin{minipage}{5.5in}
{\it Example}.
}{
\end{minipage}\\
\end{center}
}

\newenvironment{implementation}{
\begin{center}
\begin{tabular}{|c}
\begin{minipage}{5.5in}
{\it Implementation note}.
}{
\end{minipage}\\
\end{tabular}
\end{center}
}

\newcommand{\rsec}[1]
           {\S\ref{#1}}

% courtesy: http://www.iam.ubc.ca/~newbury/tex/page-set-up.html
\newcommand{\sekshun}[1]
           {
             \section{#1}
             \markboth{Chapel Language Specification}{#1}
           }

\oddsidemargin 0.0in
\evensidemargin 0.5in
\textwidth 6in
\headheight 0.2in
\topmargin 0in
\headsep 0.3in
\textheight 8.5in

\makeindex

\title{Chapel Language Specification 0.6}

\author{Cray Inc\\
411 First Ave S, Suite 600\\
Seattle, WA 98104}

\date{}

\setcounter{tocdepth}{3}

\begin{document}

\pagestyle{empty}

\maketitle

\cleardoublepage
\include{tm}
\cleardoublepage

\pagestyle{myheadings}
\markboth{Chapel Language Specification}{Chapel Language Specification}
\pagenumbering{roman}

\tableofcontents

\cleardoublepage

\pagestyle{myheadings}
\pagenumbering{arabic}

\setlength{\parindent}{0in}
\setlength{\parskip}{4mm plus2mm minus1mm}

\sekshun{Scope}

\sekshun{Notation}

\sekshun{Organization}

\sekshun{Acknowledgments}

\sekshun{Language Overview}

\subsection{Motivating Principles}
\subsubsection{Global View Programming Model}
\subsubsection{Locality Aware Programming}
\subsubsection{Object-Oriented Programming}
\subsubsection{Generic Programming}

\subsection{Basic Language Features}
\subsubsection{Programs and Modules}
\subsubsection{Data Types and Variables}
\subsubsection{Statements and Expressions}
\subsubsection{Structured Data Types}
\subsubsection{Functions and Methods}
\subsubsection{Sequences and Iterators}
\subsubsection{Arrays and Distributions}

\subsection{Parallel Features}
\subsubsection{Data Parallel Constructs}
\subsubsection{Task Parallel Constructs}
\subsubsection{Exploiting Data Locality}
\subsubsection{Synchronizing and Serializing Tasks}

\subsection{Data Distributions}

\sekshun{Lexical Structure}

\subsection{Programs}
\subsection{Comments}
\subsection{White Space}
\subsection{Case Sensitivity}
\subsection{Tokens}
\subsubsection{Identifiers}
\subsubsection{Keywords}
\subsubsection{Literals}
\subsubsection{Operators and Punctuation}
\subsubsection{Grouping Tokens}
\subsection{Compile-Time Conditionals}
\subsection{User-Defined Compiler Errors}

\sekshun{Types}

\subsection{Primitive Types}
\subsubsection{The Bool Type}
\subsubsection{Signed and Unsigned Integral Types}
\subsubsection{Floating-Point Types}
\subsubsection{Complex Types}
\subsubsection{The String Type}
\subsubsection{Primitive Type Literals}
\subsection{Enumerated Types}
\subsection{Class Types}
\subsection{Record Types}
\subsection{Union Types}
\subsection{Tuple Types}
\subsection{Sequence Types}
\subsection{Domain and Array Types}
\subsection{Type Aliases}

\sekshun{Variables}

\subsection{Variable Declarations}
\subsubsection{Default Initialization}
\subsubsection{Local Type Inference}
\subsection{Global Variables}
\subsection{Local Variables}
\subsection{Constants}
\subsubsection{Compile-Time Constants}
\subsubsection{Runtime Constants}
\subsection{Configuration Variables}
\subsection{Synchronization Variables}
\subsection{Single Variables}

\sekshun{Conversions}

\subsection{Implicit Conversions}
\subsubsection{Implicit Numeric Conversions}
\subsubsection{Implicit Enumeration Conversions}
\subsubsection{Implicit Class Conversions}
\subsubsection{Implicit Record Conversions}
\subsubsection{Implicit Compile-Time Constant Conversions}
\subsubsection{Implicit Statement Bool Conversions}
\subsection{Explicit Conversions}
\subsubsection{Explicit Numeric Conversions}
\subsubsection{Explicit Enumeration Conversions}
\subsubsection{Explicit Class Conversions}
\subsubsection{Explicit Record Conversions}

\sekshun{Expressions}

\subsection{Primary Expressions}
\subsubsection{Literal Expressions}
\subsubsection{Variable Expressions}
\subsubsection{Member Access Expressions}
\subsubsection{Call Expressions}
\subsubsection{Indexing Expressions}
\subsubsection{The Type Query Expression}
\subsubsection{Casts}
\subsection{Operators}
\subsubsection{Operator Precedence and Associativity}
\subsubsection{Operator Overloading}
\subsection{Arithmetic Operators}
\subsubsection{Unary Plus Operators}
\subsubsection{Unary Minus Operators}
\subsubsection{Addition Operators}
\subsubsection{Subtraction Operators}
\subsubsection{Multiplication Operators}
\subsubsection{Division Operators}
\subsubsection{Modulus Operators}
\subsubsection{Exponentiation Operators}
\subsection{Bitwise Operators}
\subsubsection{Bitwise Complement Operators}
\subsubsection{Bitwise And Operators}
\subsubsection{Bitwise Or Operators}
\subsubsection{Bitwise Xor Operators}
\subsection{Shift Operators}
\subsection{Logical Operators}
\subsubsection{Logical Negation Operators}
\subsubsection{Logical And Operators}
\subsubsection{Logical Or Operators}
\subsection{Relational Operators}
\subsubsection{Ordered Comparison Operators}
\subsubsection{Equality Comparison Operators}
\subsection{Miscellaneous Operators}
\subsubsection{The String Concatenation Operator}
\subsubsection{The Sequence Concatenation Operator}
\subsubsection{The Arithmetic Domain By Operator}
\subsubsection{The Sequence By Operator}
\subsection{Let Expressions}
\subsection{Conditional Expressions}
\subsection{Forall Expressions}
\subsection{Constant Expressions}

\sekshun{Statements}

\subsection{Blocks}
\subsection{Block Level Statements}
\subsection{Expression Statements}
\subsection{Assignment Statements}
\subsection{The Conditional Statement}
\subsection{The Select Statement}
\subsection{The While and Do While Loops}
\subsection{The For Loop}
\subsubsection{Zipper Iteration}
\subsubsection{Tensor Product Iteration}
\subsection{The Use Statement}
\subsection{The Type Select Statement}
\subsection{The Empty Statement}

\sekshun{Modules}

\subsection{Module Definitions}
\subsection{Program Execution}
\subsubsection{The {\em main} Function}
\subsubsection{Command-Line Arguments}
\subsection{Module Execution}
\subsubsection{The {\em initialize} Function}
\subsubsection{The {\em finalize} Function}
\subsubsection{Programs with a Single Module}
\subsection{Module Scopes}
\subsubsection{Using Modules}
\subsubsection{Explicit Naming}
\subsection{Nested Modules}
\subsection{Implicit Module Names}

\sekshun{Functions}

\subsection{Function Definitions}
\subsection{The Return Statement}
\subsection{Function Calls}
\subsection{Formal Arguments}
\subsubsection{Named Arguments}
\subsubsection{Default Values}
\subsection{Intents}
\subsubsection{The Blank Intent}
\subsubsection{The In Intent}
\subsubsection{The Out Intent}
\subsubsection{The Inout Intent}
\subsection{Variable Functions}
\subsubsection{Explicit Setter Functions}
\subsection{Function Overloading}
\subsection{Function Resolution}
\subsubsection{Identifying Visible Functions}
\subsubsection{Determining Candidate Functions}
\subsubsection{Determining More Specific Functions}
\subsubsection{Visibility and Function Resolution}
\subsubsection{Functions with Class Arguments}
\subsection{Most Specific Function}
\subsection{Nested Functions}
\subsubsection{Accessing Outer Variables}
\subsection{Variable Length Argument Lists}
\subsection{Special Functions}

\sekshun{Classes}

\subsection{Class Declarations}
\subsubsection{Class Fields}
\subsubsection{Class Field Accesses}
\subsection{Class Instances}
\subsubsection{Class Assignment}
\subsection{Class Constructors}
\subsubsection{The Default Constructor}
\subsubsection{User-Defined Constructors}
\subsubsection{Ambiguities in Constructor Calls}
\subsection{Class Methods}
\subsubsection{Class Method Declarations}
\subsubsection{The {\em this} Reference}
\subsubsection{Class Method Calls}
\subsubsection{Class Methods without Parentheses}
\subsubsection{The {\em this} Method}
\subsection{Getters and Setters}
\subsubsection{Default Getters and Setters}
\subsubsection{User-Defined Getters and Setters}
\subsection{Inheritance}
\subsubsection{Derived Class Definition}
\subsubsection{Accessing Base Class Fields}
\subsubsection{Derived Class Constructors}
\subsubsection{Shadowing Base Class Fields}
\subsubsection{Overriding Base Class Functions}
\subsubsection{Inheriting from Multiple Classes}
\subsection{Iteration over Classes}
\subsubsection{The Iteration Interface}
\subsection{Nested Type Definitions in Classes}
\subsection{Automatic Memory Management}

\sekshun{Records}

\subsection{Record Declarations}
\subsection{Class and Record Differences}
\subsubsection{Records as Value Classes}
\subsubsection{Record Inheritance}
\subsubsection{Record Assignment}

\sekshun{Unions}

\subsection{Union Declarations}
\subsubsection{Union Fields}
\subsubsection{Union Field Accesses}
\subsection{Union Assignment}
\subsection{Union Inheritance}
\subsection{The Type Select Statement and Unions}
\subsection{Record and Union Differences}

\sekshun{Tuples}

\subsection{Tuple Expressions}
\subsection{Tuple Type Definitions}
\subsection{Tuple Assignment}
\subsection{Tuple Destructuring}
\subsubsection{Variable Declarations in a Tuple}
\subsubsection{Ignoring Values with Underscore}
\subsection{Tuple Indexing}
\subsection{Homogeneous Tuples}
\subsubsection{Declaring Homogeneous Tuples}
\subsubsection{Indexing of Homogeneous Tuples}
\subsection{Formal Arguments of Tuple Type}
\subsubsection{Formal Argument Declarations in a Tuple}

\sekshun{Sequences}

\subsection{Sequence Expressions}
\subsection{Sequence Type Definitions}
\subsection{Sequence Assignment}
\subsection{Iteration over Sequences}
\subsection{Sequence Concatenation}
\subsection{Sequence Indexing}
\subsubsection{Sequence Indexing by Integers}
\subsubsection{Sequence Indexing by Tuples}
\subsection{Sequence Promotion of Scalar Functions}
\subsubsection{Zipper Promotion}
\subsubsection{Tensor Product Promotion}
\subsection{Sequence Equality}
\subsection{Sequences in Logical Contexts}
\subsubsection{Sequences in Conditional Statements}
\subsubsection{Sequences in While and Do While Loops}
\subsubsection{Sequences in Select Statements}
\subsection{Filtering Predicates}
\subsection{Methods and Functions on Sequences}
\subsubsection{The {\em length} Method}
\subsubsection{The {\em reverse} Method}
\subsubsection{The {\em spread} Function}
\subsubsection{The {\em transpose} Function}
\subsubsection{The {\em reshape} Function}
\subsection{Arithmetic Sequences}
\subsubsection{Strided Arithmetic Sequences}
\subsubsection{Indefinite Sequences}
\subsubsection{Indexing into Strings with Arithmetic Sequences}
\subsection{Sequences of One-Character Strings}
\subsection{Conversions Between Sequences and Tuples}

\sekshun{Domains and Arrays}

\subsection{Domains}
\subsubsection{Domain Types}
\subsubsection{Domain Assignment}
\subsubsection{Formal Arguments of Domain Type}
\subsubsection{Iteration over Domains}
\subsection{Arrays}
\subsubsection{Array Types}
\subsubsection{Array Indexing}
\subsubsection{Array Assignment}
\subsubsection{Formal Arguments of Array Type}
\subsubsection{Iteration over Arrays}
\subsubsection{Array Initialization}
\subsection{Index Types}
\subsubsection{Index Methods on Domains}
\subsection{Arithmetic Domains and Arrays}
\subsubsection{Arithmetic Domains and Arithmetic Sequences}
\subsubsection{Multidimensional Arithmetic Domains and Arrays}
\subsubsection{Strided Arithmetic Domains and Arrays}
\subsubsection{Arithmetic Domain and Array Types}
\subsubsection{Arithmetic Domain Index Types}
\subsubsection{Arithmetic Array Indexing}
\subsubsection{Arithmetic Domain Indexing}
\subsubsection{Formal Arguments of Arithmetic Array Type}
\subsubsection{Methods on Arithmetic Domains and Arrays}
\subsection{Sparse Arithmetic Domains and Arrays}
\subsubsection{Adding Indices to Sparse Arithmetic Domains}
\subsubsection{Removing Indices from Sparse Arithmetic Domains}
\subsection{Indefinite Domains and Arrays}
\subsubsection{Indefinite Domain and Array Types}
\subsubsection{Indefinite Domain Index Types}
\subsubsection{Adding Indices to Indefinite Domains}
\subsubsection{Removing Indices from Indefinite Domains}
\subsubsection{Methods on Indefinite Domains and Arrays}
\subsection{Opaque Domains and Arrays}
\subsubsection{Opaque Domain and Array Types}
\subsubsection{Opaque Domain Index Types}
\subsubsection{Adding Indices to Opaque Domains}
\subsubsection{Removing Indices from Opaque Domains}
\subsubsection{Methods on Opaque Domains and Arrays}
\subsection{Enumerated Domains and Arrays}
\subsubsection{Enumerated Domain and Array Types}
\subsubsection{Enumerated Domain Index Types}
\subsubsection{Methods on Enumerated Domains and Arrays}
\subsection{Product Domains and Arrays}
\subsubsection{Product Domain Definition}
\subsubsection{Methods on Product Domains and Arrays}
\subsubsection{Arrays of Arrays}
\subsection{Association of Arrays to Domains}
\subsubsection{Preservative Reallocation of Arrays}
\subsubsection{Destructive Reallocation of Arrays}
\subsection{Subdomains}
\subsubsection{Subdomain Definition}
\subsubsection{Index Methods on Subdomains}
\subsubsection{Association of Subdomains to Domains}
\subsection{Anonymous Domains}
\subsection{Array Promotion of Scalar Functions}
\subsection{Array Slicing}

\sekshun{Iterators}

\subsection{Iterator Functions}
\subsection{The Yield Statement}
\subsection{Iterator Calls}
\subsubsection{Iterators in For and Forall Loops}
\subsubsection{Iterators as Sequences}
\subsection{Making a Class an Iterator}
\subsubsection{The {\em rank} Parameter}
\subsubsection{The {\em getHeadCursor} Method}
\subsubsection{The {\em getNextCursor} Method}
\subsubsection{The {\em getValue} Method}
\subsubsection{The {\em isValidCursor?} Method}

\sekshun{Generics}

\subsection{Generic Functions}
\subsubsection{Formal Arguments of Generic Type}
\subsubsection{Formal Arguments without Types}
\subsubsection{Formal Arguments with Queried Types}
\subsubsection{Generic Array Argument Types}
\subsubsection{Generic Domain and Sequence Types}
\subsection{Instantiating Generic Functions}
\subsubsection{Function Resolution and Generic Functions}
\subsubsection{Function Visibility in Generic Functions}
\subsection{Generic Types}
\subsubsection{Generic Default Constructors}
\subsubsection{Generic Methods}
\subsubsection{Type Aliases and Generic Types}
\subsubsection{The {\em elt\_type} Type}
\subsection{Parameters and Generics}
\subsubsection{Instantiating Parameterized Generics}
\subsection{Where Expressions}
\subsubsection{Parameter Expressions}
\subsubsection{Type Expressions}

\sekshun{Parallelism and Synchronization}

\subsection{Parallel Execution}
\subsubsection{Sequence}
\subsubsection{Scan and Reduce}
\subsubsection{Parallel Array Assignment}
\subsubsection{Ordered Expressions}
\subsection{Parallel Statements}
\subsubsection{Forall}
\subsubsection{Alternative Forall Loop Syntax}
\subsubsection{Ordered Forall}
\subsubsection{Cobegin}
\subsubsection{Begin}
\subsubsection{Serial}
\subsection{Parallel Iterators}
\subsection{Implicit Synchronization on Variables}
\subsubsection{Single Assignment Variables}
\subsubsection{Sync Variables}
\subsubsection{Synchronization Variables of Record Type}
\subsubsection{Synchronization Variables of Class Type}
\subsubsection{Functions on Synchronization Variables}
\subsection{Memory Consistency}
\subsection{Atomic Transactions}
\subsubsection{Strong Atomicity}
\subsubsection{Weak Atomicity}

\sekshun{Locality and Distribution}

\subsection{Locales}
\subsubsection{The Locale Type}
\subsubsection{Parallel Execution Model}
\subsubsection{Predefined Locales Array}
\subsubsection{Querying the Locale of a Variable}
\subsection{Specifying Locales for Computation}
\subsubsection{On}
\subsubsection{On and Forall Loops}
\subsubsection{On and Iterators}
\subsection{Distributions}
\subsubsection{Distributed Domains}
\subsubsection{Distributed Arrays}
\subsubsection{Undistributed Domains and Arrays}
\subsection{Standard Distributions}
\subsubsection{Block Distribution}
\subsubsection{Cyclic Distribution}
\subsubsection{BlockCyclic Distribution}
\subsubsection{Cut Distribution}
\subsection{User-Defined Distributions}

\sekshun{Reductions and Scans}

\sekshun{Input and Output}

\sekshun{Standard Modules}

\subsection{Math}
\subsection{System}
\subsection{Bitwise Functions}
\subsection{Time}
\subsection{Random Numbers}

\NeedsTeXFormat{LaTeX2e}
\ProvidesClass{spec}

\DeclareOption*{\PassOptionsToClass{\CurrentOption}{book}}

\ProcessOptions

\LoadClass{book}

\renewcommand{\@makechapterhead}[1]{%
  { \parindent \z@ \raggedright 
    \normalfont \huge\bfseries
    \ifnum \c@secnumdepth >\m@ne
      \if@mainmatter
        \thechapter
		\space\space\space\space
      \fi
    \fi
    #1\par\nobreak
    \vskip 40\p@
  }
}

\renewcommand{\@makeschapterhead}[1]{%
  {\parindent \z@ \raggedright
    \normalfont \huge \bfseries
    #1\par\nobreak
    \vskip 40\p@
  }
}

\endinput


\end{document}
