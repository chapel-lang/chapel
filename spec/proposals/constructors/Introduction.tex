\section{Introduction}

The original design for Chapel anticpated that memory management would be handled
by garbage collection.  However, efficient distributed-memory garbage collection
algorithms have not been forthcoming.  In the mean time, the desire to support a growing
number of users by keeping memory leakage within practical bounds forced the
implementation to provide some means for reclaiming memory.  

Since memory reclamation was not part of the original design, it is necessarily somewhat
ad hoc.  A review of the memory management elements in use in the Chapel compiler revealed
that all of the necessary elements are present, but have not been applied uniformly across
all types.  As a result, memory allocated for use by certain types still leaks.

By comparison with C++, which has a strong User-Managed Memory model, the Chapel
implementation still lacks some elements that would support memory management under
program control --- at least in a more unified manner.  This proposal restates the object
model as it relates to the current Chapel implementation, and provides recommendations for
syntax and semantics modifications necessary to support that model.

The remainder of this document is organized as follows: Section \rsec{Objects} discusses
the object lifecycle and provides a brief overview of the User-Managed Memory model.  It
also contains general information common to the other sections.  Secion
\rsec{Declarations} presents the proposed syntax and semantics for variable and field
declarations outside the context of a constructor definition, highlighting how this
differs from the current implementation.  Section \rsec{Constructors} presents the
proposed syntax and semantics for constructors, highlighting how this differs from the current impelemtation.

Appendices are provided for quick reference.  Appendix \rsec{Changes} provides a summary of the
proposed changes.  Appendix \rsec{Examples} gives examples including declarations for the
fundamental types provided by Chapel.

