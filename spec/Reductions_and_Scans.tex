\sekshun{Reductions and Scans}
\label{Reductions_and_Scans}

Chapel provides a set of built-in reductions and scans with parallel
semantics, a mechanism for defining more reductions and scans with
efficient implementations, and syntact support to make reductions and
scans easy to use.

\subsection{Reduction Expressions}
\label{reduce}

The syntax for a reduction expression is given by:
\begin{syntax}
reduce-expression:
  reduce-operator `reduce' expression
  type `reduce' expression

reduce-scan-operator: one of
  + * && || & | ^ min max
\end{syntax}

The expression on the right-hand side of the reduction can be of any
type that can be iterated over.

The built-in reductions are defined in \sntx{reduce-scan-operator}.  These
include, in order, sum, product, logical and, logical or, bitwise and,
bitwise or, bitwise exclusive or, minimum, and maximum.

User-defined reductions are specified by preceding the
keyword \chpl{reduce} by the class type that implements the reduction
interface as described in~\rsec{udr}.

\subsection{Scan Expressions}
\label{scan}

The syntax for a scan expression is given by:
\begin{syntax}
scan-expression:
  reduce-scan-operator `scan' expression
  type `scan' expression
\end{syntax}

The expression on the right-hand side of the scan can be of any
type that can be iterated over.

The built-in scans are defined in \sntx{reduce-scan-operator}.  These
are identical to the built-in reductions and are described
in~\rsec{reduce}.

User-defined scans are specified by preceding the keyword \chpl{scan}
by the class type that implements the scan interface as described
in~\rsec{udr}.

\subsection{User-Defined Reductions and Scans}
\label{udr}

User-defined reductions and scans are supported via class definitions
where the class implements a structural interface.  The definition of
this structural interface is forthcoming.  The following paper
sketched out such an interface:
\begin{quote}
S.~J.~Deitz, D.~Callahan, B.~L.~Chamberlain, and L.~Snyder.  {\bf
Global-view abstractions for user-defined reductions and scans}.  In
{\it Proceedings of the Eleventh ACM SIGPLAN Symposium on Principles
and Practice of Parallel Programming}, 2006.
\end{quote}
