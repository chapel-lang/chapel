The \chpl{ordered} keyword can be used as an unary operator to
suppress parallel execution of a sequence expression that can involve
side-effects to memory. The \chpl{ordered} keyword does not inhibit
parallelism within the sub-expression.

\begin{example}
\begin{chapel}
ordered [i in S] f(i) 
\end{chapel}
f is a function and S is a sequence. Each instance of
f(S$_i$) is executed one at a time and in sequence order. The
\chpl{ordered} constraint does not propagate to inhibit parallelism
within f.
\end{example}
