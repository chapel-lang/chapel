A synchronized variable generalizes the single assignment variable to
permit multiple assignments over its lifetime. Synchronized variables
are declared with the type attribute \chpl{sync}. The synchronized
variable syntax is
\begin{chapel}
[[var]] <symbol>: sync <type> [= <expr>];
\end{chapel}
A sync variable is logically either {\em full} or {\em empty}. When it
is empty, computations that attempt to read that variable are delayed
until it becomes full by the next assignment to it which atomically
changes the state to full. When the variable is full, a variable
reference consumes the value and atomically transitions the state back
to empty. If there are more than one computation waiting, one is
non-deterministically selected to receive the value resume execution.
The other computations continue to wait for the next
assignment. Otherwise, the state transistions to empty. 

If a computation attempts to assign to a synchronized variable that is
full, the computation is suspended and the assignment is delayed. When
the variable becomes empty, the computation is resumed and the
assignment proceeds, transistioning the state back to full. If there
are multiple computations attempting such an assignment, one is
non-deterministically selected to proceed and the other computations
continue to wait until the sync variable is emptied again.

\chpl{sync} variables allow a sequence of values to be communicated
between computations using a single shared variable. They also can be
used as building blocks for more traditional synchronization
primitives such as semaphores and monitors.
