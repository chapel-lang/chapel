The first form of creating parallelism is with the \chpl{forall}
statement. The \chpl{forall} statement is a variant of \chpl{for} that 
enables concurrent execution. The \chpl{forall} statement syntax is
\begin{chapel}
[ordered] forall <var> in <expr> <block>
\end{chapel}
Here \chpl{<var>} is either a symbol or a tuple of symbols and
\chpl{<expr>} evaluates to a sequence of suitable type. This
statement evaluates \chpl{<block>} once for each element in the
sequence. The evaluation of each of these statements can be executed
concurrently and is considered a separate computation. Control
continues with the statement following the \chpl{forall} only after
all statement instances have been completely evaluated.

Control transfers such as \chpl{goto}, \chpl{break}, \chpl{continue},
and \chpl{return} are not permitted either into or out of the body of
a \chpl{forall} statement.  However, \chpl{yield} statements are permitted.
