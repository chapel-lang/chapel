\noindent @@@\hrulefill \\
The cobegin statement spawns a computation for each of the statements
in the block-statement that it immediately precedes.  When each of
these computations finish, control continues with the next statement.
The syntax for the cobegin statement is given by
\begin{syntax}
cobegin-statement:
  `cobegin' block-statement
\end{syntax}
\noindent @@@\hrulefill \\

The second form of creating parallelism is with the \chpl{cobegin}
statement. The \chpl{cobegin} statement syntax is
\begin{chapel}
cobegin <compound statement> 
\end{chapel}
Each statement in the \chpl{<compound\ statement>} is executed 
concurrently with every other statement and is considered a separate 
computation.

Control continues after all of these statements within the compound
statement have been evaluated. As with \chpl{forall}, control
transfers are not permitted either into or out of the body of a
\chpl{cobegin} statement. As with \chpl{forall}, \chpl{yield}
statements are allowed.

Variables declared in the \chpl{cobegin} compound statment are {\em single
assignment variables} \ref{Single_Assignment_Variables}. 
