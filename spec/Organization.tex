\sekshun{Organization}
\label{Organization}

This specification is organized as follows:
\begin{itemize}

\item
Section~\ref{Scope}, Scope, describes the scope of this specification.

\item
Section~\ref{Notation}, Notation, introduces the notation that is used
throughout this specification.

\item
Section~\ref{Organization}, Organization, describes the contents of
each of the sections within this specification.

\item
Section~\ref{Acknowledgments}, Acknowledgments, offers a note of
thanks to people and projects.

\item
Section~\ref{Language_Overview}, Language Overview, describes Chapel
at a high level.

\item
Section~\ref{Lexical_Structure}, Lexical Structure, describes the
lexical components of Chapel.

\item
Section~\ref{Types}, Types, describes the types in Chapel and defines
the primitive and enumerated types.

\item
Section~\ref{Variables}, Variables, describes variables and constants
in Chapel.

\item
Section~\ref{Conversions}, Conversions, describes the legal implicit
and explicit conversions allowed between values of different types.
Chapel does not allow for user-defined conversions.

\item
Section~\ref{Expressions}, Expressions, describes the serial
expressions in Chapel.

\item
Section~\ref{Statements}, Statements, describes the serial statements
in Chapel.

\item
Section~\ref{Modules}, Modules, describes modules, Chapel's
abstraction to allow for name space management.

\item
Section~\ref{Functions}, Functions, describes functions and function
resolution in Chapel.

\item
Section~\ref{Tuples}, Tuples, describes tuples in Chapel.

\item
Section~\ref{Classes}, Classes, describes reference classes in Chapel.

\item
Section~\ref{Records}, Records, describes records or value classes in
Chapel.

\item
Section~\ref{Unions}, Unions, describes unions in Chapel.

\item
Section~\ref{Ranges}, Ranges, describes ranges in Chapel.

\item
Section~\ref{Domains_and_Arrays}, Domains and Arrays, describes
domains and arrays in Chapel.  Chapel arrays are more general than
arrays in many other languages.  Domains are index sets, an
abstraction that is typically not distinguished from arrays.

\item
Section~\ref{Iterators}, Iterators, describes iterator functions and
promotion.

\item
Section~\ref{Generics}, Generics, describes Chapel's support for
generic functions and types.

\item
Section~\ref{Task_Parallelism_and_Synchronization}, Task Parallelism
and Synchronization, describes task-parallel expressions and
statements in Chapel as well as synchronization constructs, atomic
sections, and memory consistency.

\item
Section~\ref{Data_Parallelism}, Data Parallelism, describes
data-parallel expressions and statements in Chapel including
reductions and scans.

\item
Section~\ref{User_Defined_Reductions_and_Scans}, User-Defined
Reductions and Scans, describes how Chapel programmers can define
their own reduction and scan operators.

\item
Section~\ref{Locales}, Locales, describes constructs for managing
locality and executing Chapel programs on distributed-memory systems.

\item
Section~\ref{Domain_Maps}, Domain Maps, describes
Chapel's \emph{domain map} construct for defining the layout of
domains and arrays within a single locale and/or the distribution of
domains and arrays across multiple locales.

\item
Section~\ref{User_Defined_Domain_Maps}, User-Defined Domain Maps,
describes how Chapel programmers can define their own domain maps to
implement domains and arrays.

\item
Section~\ref{Input_and_Output}, Input and Output, describes support
for input and output in Chapel, including file input and output..

\item
Section~\ref{Standard_Distributions}, Standard Distributions,
describes the standard distributions (multi-locale domain maps) that
are provided with the Chapel language.

\item
Section~\ref{Standard_Layouts}, Standard Layouts, describes the
standard layouts (single locale domain maps) that are provided with
the Chapel language.

\item
Section~\ref{Standard_Modules}, Standard Modules, describes the
standard modules that are provided with the Chapel language.

\end{itemize}
