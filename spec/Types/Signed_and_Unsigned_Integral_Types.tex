The integral types can be parameterized by the number of bits used to
represent them.  The default signed integral type, \chpl{int}, and the
default unsigned integral type, \chpl{uint}, are 32 bits.

The integral types and their ranges are given in the following table:

\begin{center}
\begin{tabular}{|l|r|r|}
\hline
{\bf Type} & {\bf Minimum Value} & {\bf Maximum Value} \\
\hline
{\tt int(8)} & -128 & 127 \\
{\tt uint(8)} & 0 & 255 \\
{\tt int(16)} & -32768 & 127 \\
{\tt uint(16)} & 0 & 65535 \\
{\tt int(32)}, {\tt int} & -2147483648 & 2147483647 \\
{\tt uint(32)}, {\tt uint} & 0 & 4294967295 \\
{\tt int(64)} & -9223372036854775808 & 9223372036854775807 \\
{\tt uint(64)} & 0 & 18446744073709551615 \\
\hline
\end{tabular}
\end{center}

The unary and binary operators that are pre-defined over the integral
types operate with 32- and 64-bit precision.  Using these operators on
integral types represented with fewer bits results in a coercion
according to the rules defined in Section~\ref{Implicit_Conversions}.
