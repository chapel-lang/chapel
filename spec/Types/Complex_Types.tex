Like the integral and real types, the complex types can be
parameterized by the number of bits used to represent them.  A complex
number is composed of two real numbers so the number of bits used to
represent a complex is twice the number of bits used to represent the
real numbers.  The default complex type, \chpl{complex}, is 128 bits.
The complex types that are supported are machine-dependent, but
usually include \chpl{complex(64)} and \chpl{complex(128)}, and
sometimes include \chpl{complex(256)}.

The real and imaginary components can be accessed via the methods
\chpl{re} and \chpl{im}.  The type of these components is real.

\begin{example}
Given a complex number \chpl{3.14+2.72i}, the expressions
\chpl{c.re} and \chpl{c.im} refer to \chpl{3.14} and \chpl{2.72}
respectively.
\end{example}
