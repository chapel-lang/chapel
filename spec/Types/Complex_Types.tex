Like the integral and floating-point types, the complex types can be
parameterized by the number of bits used to represent them.  A complex
number is composed of two floating-point numbers so the number of bits
used to represent a complex is twice the number of bits used to
represent the floating-point numbers.

The real and imaginary components can be accessed via the methods
\chpl{real} and \chpl{imag}.

\begin{example}
Given a complex number \chpl{3.14+2.72i}, the expressions
\chpl{c.real} and \chpl{c.imag} refer to \chpl{3.14} and \chpl{2.72}
respectively.
\end{example}
