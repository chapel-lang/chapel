Bool literals are designated by the following syntax:
\begin{syntax}
bool-literal: one of
  `true' `false'
\end{syntax}

Signed and unsigned integer literals are designated by the following
syntax:
\begin{syntax}
integer-literal:
  digits

digits:
  digit
  digit digits

digit: one of
  0 1 2 3 4 5 6 7 8 9
\end{syntax}
Suffixes, like those in C, are not necessary.  The type of an integer
literal is the first type of the following that can hold the value of
the digits: \chpl{int}, \chpl{int(64)}, \chpl{uint(64)}.  Explicit
conversions are necessary to change the type of the literal to another
integer size.

Floating-point literals are designated by the following syntax:
\begin{syntax}
floating-point-literal:
  digits[OPT] . digit digits[OPT] exponent-part[OPT]

exponent-part:
  `e' sign[OPT] digits

sign: one of
  + -
\end{syntax}
The type of a floating-point literal is \chpl{float}.  Explicit
conversions are necessary to change the type of the literal to another
floating-point size.

Note that floating-point literals require that a digit follow the
decimal point.  This is necessary to avoid an ambiguity in
interpreting \chpl{2.e+2} that arises if a method called \chpl{e} is
defined on integers.

Complex literals are designated by the following syntax:
\begin{syntax}
complex-literal:
  floating-point-literal `i'
  integer-literal `i'
\end{syntax}
This complex-literal form specifies the imaginary part.  The real part
can be specified in a separate integer or floating-point literal and
then added to the imaginary part.

Alternatively, a 2-tuple literal of expressions of integer or
floating-point type can be cast to a complex.  These expressions can
be literals, but do not need to be.  To create a complex literal or
parameter, they must be literals or parameters.

\begin{example}
The following codes represent the same complex literal:
\begin{center}
\chpl{2.0i}, \hspace{1pc} \chpl{0.0+2.0i}, \hspace{1pc}
\chpl{(0.0,2.0):complex}.
\end{center}
\end{example}

String literals are designated by the following syntax:
\begin{syntax}
string-literal:
  " characters "
  ' characters '

characters:
  character
  character characters

character:
  any-character
\end{syntax}

\begin{implementation}
Strings are currently restricted to ASCII characters.  In a future
version of Chapel, strings will be defined over alphabets to allow for
more exotic characters.
\end{implementation}
