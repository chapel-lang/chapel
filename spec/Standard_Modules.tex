\sekshun{Standard Modules}
\label{Standard_Modules}

This section describes the functions and types defined in the standard
modules that can be used by any Chapel program.  These modules include
the following:
\begin{itemize}
\item BitOps
\item Math
\item Random
\item Time
\end{itemize}
There is an expectation that each of these modules will be extended
and that more standard modules will be defined.

\subsection{BitOps}
\label{BitOps}

The module \chpl{BitOps} defines routines that manipulate the bits of
values of integral types.

\vspace{1pc}

\begin{protohead}
def bitPop(i: integral): int
\end{protohead}
\begin{protobody}
Returns the number of bits set to one in the integral
argument \chpl{i}.
\end{protobody}

\begin{protohead}
def bitMatMultOr(i: uint(64), j: uint(64)): uint(64)
\end{protohead}
\begin{protobody}
Returns the bitwise matrix multiplication of \chpl{i} and \chpl{j}
where the values of \chpl{uint(64)} type are treated as $8 \times 8$
bit matrices and the combinator function is bitwise or.
\end{protobody}

\begin{protohead}
def bitRotLeft(i: integral, shift: integral): i.type
\end{protohead}
\begin{protobody}
Returns the value of the integral argument \chpl{i} after rotating the
bits to the left \chpl{shift} number of times.
\end{protobody}

\begin{protohead}
def bitRotRight(i: integral, shift: integral): i.type
\end{protohead}
\begin{protobody}
Returns the value of the integral argument \chpl{i} after rotating the
bits to the right \chpl{shift} number of times.
\end{protobody}

\subsection{Math}
\label{Math}

This section is forthcoming.

\subsection{Random}
\label{Random}

This section is forthcoming.

\subsection{Time}
\label{Time}

The module \chpl{Time} defines routines that query the system time and
a record \chpl{Timer} that is useful for timing portions of code.

\vspace{1pc}

\begin{protohead}
record Timer
\end{protohead}
\begin{protobody}
A timer is used to time portions of code.  Its semantics are similar
to a stopwatch.
\end{protobody}

\begin{protohead}
enum TimeUnits { microseconds, milliseconds, seconds, minutes, hours };
\end{protohead}
\begin{protobody}
The enumeration TimeUnits defines units of time.  These units can be
supplied to routines in this module to specify the desired time units.
\end{protobody}

\begin{protohead}
def getCurrentDate(): (int, int, int)
\end{protohead}
\begin{protobody}
Returns the year, month, and day of the month as integers.  The year
is the year since 0.  The month is in the range 1 to 12.  The day is
in the range 1 to 31.
\end{protobody}

\begin{protohead}
def getCurrentTime(unit: TimeUnits = seconds): real
\end{protohead}
\begin{protobody}
Returns the elapsed time since midnight in the units specified.
\end{protobody}

\begin{protohead}
def Timer.clear()
\end{protohead}
\begin{protobody}
Clears the elapsed time stored in the Timer.
\end{protobody}

\begin{protohead}
def Timer.start()
\end{protohead}
\begin{protobody}
Start the timer.  It is an error to start a timer that is already
running.
\end{protobody}

\begin{protohead}
def Timer.stop()
\end{protohead}
\begin{protobody}
Stops the timer.  It is an error to stop a timer that is not running.
\end{protobody}

\begin{protohead}
def Timer.elapsed(unit: TimeUnits = seconds): real
\end{protohead}
\begin{protobody}
Returns the cumulative elapsed time, in the units specified, between
calls to \chpl{start} and \chpl{stop}.  If the timer is running, the
elapsed time since the last call to \chpl{start} is added to the
return value.
\end{protobody}
