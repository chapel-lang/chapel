\sekshun{Standard Modules}
\label{Standard_Modules}

This is a stub.  This portion of the document does not exist.

\subsection{Math}
\label{Math}

This is a stub.  This portion of the document does not exist.

\subsection{System}
\label{System}

This is a stub.  This portion of the document does not exist.

\subsection{Bitwise Functions}
\label{Bitwise_Functions}

This is a stub.  This portion of the document does not exist.

\subsection{Time}
\label{Time}

The module \chpl{Time} defines routines that query the system time and
a record \chpl{Timer} that is useful for timing portions of code.

{\bf Units of time}

\begin{chapel}
enum TimeUnits { microseconds, milliseconds, seconds, minutes, hours };
\end{chapel}
The enumeration TimeUnits defines units of time.  These units can be
supplied to routines in this module to specify the desired time units.

{\bf Functions for querying the system time and date}

\begin{chapel}
def getCurrentTime(unit: TimeUnits = seconds): real
\end{chapel}
Returns the elapsed time since midnight in the units specified.

\begin{chapel}
def getCurrentDate(): (int, int, int)
\end{chapel}
Returns the year, month, and day of the month as integers.  The year
is the year since 0.  The month is in the range 1 to 12.  The day is
in the range 1 to 31.

{\bf The \chpl{Timer} record and its methods}

\begin{chapel}
record Timer
\end{chapel}
A timer is used to time portions of code.  Its semantics are similar
to a stopwatch.

\begin{chapel}
def Timer.clear()
\end{chapel}
Clears the elapsed time stored in the Timer.

\begin{chapel}
def Timer.start()
\end{chapel}
Start the timer.  It is an error to start a timer that is already
running.

\begin{chapel}
def Timer.stop()
\end{chapel}
Stops the timer.  It is an error to stop a timer that is not running.

\begin{chapel}
def Timer.elapsed(unit: TimeUnits = seconds): real
\end{chapel}
Returns the cumulative elapsed time, in the units specified, between
calls to \chpl{start} and \chpl{stop}.  If the timer is running, the
elapsed time since the last call to \chpl{start} is added to the
return value.

\subsection{Random Numbers}
\label{Random_Numbers}

This is a stub.  This portion of the document does not exist.
