\sekshun{Domain Maps}
\label{Domain_Maps}
\index{domain maps}

\emph{Domain maps} specify the implementation of domains and, in turn,
arrays by defining the mapping from indices in domains to memory
locations within or across locales.  The term \emph{layout} is used to
describe a domain map that describes domains and arrays that exist on
a single locale.  The term \emph{distribution} is used to describe a
domain map that describes domains and arrays that are partitioned
across multiple locales.

The domain map abstraction is not only used to define this mapping,
but rather is used to define the implementation of domains and arrays
including their accessors and iterators.

\subsection{Domain Map Types}
\label{Domain_Map_Types}

Domain map types are defined by the type of the implementing domain
class, but are distinct from the class type.  Typically, the domain
map class type is only used on its own in defining the domain map
itself.  Defining a domain map is discussed
in~\rsec{User_Defined_Domain_Maps}.

Specifying a domain map type involves specifying a domain map class
type and wrapping it by the domain map specifier \chpl{dmap}.
\begin{example}
The code
\begin{chapel}
use BlockDist;
var MyBlockDist: dmap(Block(rank=2));
\end{chapel}
creates a uninitialized two-dimensional Block distribution
called \chpl{MyBlockDist} that can be used to distribute 2-dimensional
arithmetic domains.  The Block distribution is described in more
detail in~\rsec{Block_Dist}.
\end{example}

\subsection{Domain Map Values}
\label{Domain_Map_Values}

Constructing a domain map value involves calling the constructor of a
domain map class and defining a new domain map type \chpl{dmap}.
\begin{example}
The code
\begin{chapel}
use BlockDist;
var MyBlockDist: dmap(Block(rank=2)) = new dmap(new Block([1..n,1..n]));
\end{chapel}
creates an initialized two-dimensional Block distribution with a
bounding box of \chpl{[1..n, 1..n]} over all of the locales.  The
Block distribution is described in more detail in~\rsec{Block_Dist}.
\end{example}

\subsection{Mapped Domains and Arrays}
\label{Mapped_Domains_and_Arrays}

\index{domains!mapped}
A domain for which a domain map is specified is referred to as a {\em
mapped domain}.

The syntax to create a mapped domain type is the same as the syntax to
create a mapped domain value:
\begin{syntax}
mapped-domain-type:
  domain-type `dmapped' domain-map-expression

mapped-domain-expression:
  domain-expression `dmapped' domain-map-expression

domain-map-expression:
  expression
\end{syntax}

\begin{example}
The code
\begin{chapel}
use BlockDist;
var MyBlockDist = new dmap(new Block([1..n,1..n]));
var Dom: domain(2) dmapped MyBlockDist = [1..n, 1..n];
\end{chapel}
defines a new Block-distributed domain, mapped via \chpl{MyBlockDist}.
\end{example}

When defining a new domain map inline with the \chpl{dmapped} keyword,
a syntactic sugar is supported in which the ``new dmap(new''
characters (along with the closing parenthesis) may be omitted.
\begin{example}
The code
\begin{chapel}
use BlockDist;
var D = [1..n, 1..n] dmapped new dmap(new Block([1..n,1..n]));
\end{chapel}
is equivalent to
\begin{chapel}
use BlockDist;
var D = [1..n, 1..n] dmapped Block([1..n,1..n]);
\end{chapel}
\end{example}

\subsection{Default Mapped Domains and Arrays}
\label{Default_Mapped_Domains_and_Arrays}

If a domain is not mapped via the \chpl{dmapped} keyword, it is
implicitly mapped by a default layout to the locale on which it is
declared.
