%
% These are special commands for adding extra information about code
% snippets which can be extracted and used to generated working
% example codes.
%


%
% Proposed structure:
% - The use of a 'chapelpre' or 'chapel' or 'chapelcode' environment
%   marks the start of a new test
%
% - New tests can be named in the 'chapelpre' enviroments via a latex
%   style comment as the first line
%
% - The use of a 'chapeloutput' or 'printchapeloutput' marks the end
%   of the test
%
% - The compopts and execopts are listed in order (one set per line)
%
% - Multiple .good files can be specified, and can be named via a
%   latex style comment as the first line
%


%
% Gobble up the text in this new box.  The text in each environment is
% dropped on the floor during latex compilation.
%
\newsavebox{\teststuff}

%
% Any additional lines needed for the code snippet to run/compile
% (before and after the chapel code segment)
%
\newenvironment{chapelpre} {\begin{lrbox}{\teststuff}
\begin{minipage}{6in}}
{\end{minipage}\end{lrbox}}

\newenvironment{chapelpost} {\begin{lrbox}{\teststuff}
\begin{minipage}{6in}}
{\end{minipage}\end{lrbox}}


%
% .compopts file
%
\newenvironment{chapelcompopts} {\begin{lrbox}{\teststuff}
\begin{minipage}{6in}}
{\end{minipage}\end{lrbox}}

%
% .execopts file
%
\newenvironment{chapelexecopts} {\begin{lrbox}{\teststuff}
\begin{minipage}{6in}}
{\end{minipage}\end{lrbox}}


%
% .good file
% To get more than one file, use a latex style comment to name the
% .good file
%
\newenvironment{chapeloutput} {\begin{lrbox}{\teststuff}
\begin{minipage}{6in}}
{\end{minipage}\end{lrbox}}


%
% .good file that is printed in the text of the Spec
% To get more than one file, use a latex style comment to name the
% .good file
%
\newenvironment{printchapeloutput} {
\begin{quote}
}{
\end{quote}
}
